% ---------------------------- Problem 1----------------------------------
\subsubsection*{\center Задача № 1.}
{\bf Условие.~}
Разложить в ряд Фурье заданную функцию $f(x)$, построить графики $f(x)$ и суммы ее ряда Фурье. Если не указывается, какой вид разложения в ряд необходимо представить, то требуется разложить функцию либо в общий тригонометрический ряд Фурье, либо следует выбрать оптимальный вид разложения в зависимости от данной функции.


\[ 
f(x)=\frac{x-\pi}{2} \text{ на отрезке } [0; \pi] \text{ по синусам кратных дуг.}
\]
{\bf Решение.~}	
%График
\begin{center}
	\begin{tikzpicture}
	\begin{axis}[xmin=-1,	xmax=4, 	ymin=-0.5,	ymax=0,
	width=0.5\textwidth,
	height=0.4\textwidth,
	axis x line=middle,
	axis y line=middle, 
	every axis x label/.style={at={(current axis.right of origin)},anchor=west},
	every inner x axis line/.append style={|-latex'},
	every inner y axis line/.append style={|-latex'},
	minor tick num=1,			
	axis equal=true,
	xlabel=$x$, 
	ylabel=$y$,          
	samples=100,
	clip=true,
	]
	\addplot[color=black, line width=1.5pt,domain=0:3.14] {(x-3.14)/2};
	\end{axis}
	\end{tikzpicture}
\end{center}
\noindent
Построим тригонометрический ряд Фурье для нечётных функций вида
$$
f(x)=\sum_{n=1}^\infty 
	\left(b_n\sin{(n\omega x)}\right),\quad\text{где}\,\omega=\frac{2\pi}{T},\,T=2\pi.
$$
\noindent
Вычислим коэффициенты
$$
\begin{array}{rcl}
b_n &=& \displaystyle\frac{2}{\pi}
	\int\limits_0^\pi
	\frac{x-\pi}{2} sin(nx)dx ={}									\\[12pt]
	&=& \displaystyle\frac{1}{\pi n}\left(-\left.x\cos(nx)+\int cos(nx)dx\right) \right|_0^\pi
	+\left.\frac{1}{n}\cos(xn) \right|_0^\pi = 	\\[12pt]
	&=& \displaystyle\frac{-\pi cos(\pi n) + \frac{sin(\pi n)}{n} + 0 cos(0) - \frac{sin(0)}{n}}{2\pi n}+\frac{cos(\pi n) - cos(0)}{n} = \frac{(-1)^n - 1}{n} - \frac{(-1)^n}{n},	\\[12pt]
\end{array}
$$
Применив теорему Дирихле о поточечной сходимости ряда Фурье, видим, что построенный ряд Фурье сходится 
к периодическому (с периодом $T=2\pi$) продолжению исходной функции при всех $x\ne 2\pi n$, и 
$S(2\pi n)= 0$ при $n=0,\pm1,\pm2,\ldots$, где $S(x)$ --- сумма ряда Фурье. 
График функции $S(x)$ имеет следующий вид
\begin{center}
	\begin{tikzpicture}
	\begin{axis}[xmin=-9, xmax=9, ymin=-1, ymax=1.5,
	width=0.8\textwidth,
	height=0.4\textwidth,
	axis x line=middle,
	axis y line=middle, 
	every axis x label/.style={at={(current axis.right of origin)},anchor=west},
	every inner x axis line/.append style={|-latex'},
	every inner y axis line/.append style={|-latex'},
	minor tick num=1,			
	axis equal=true,
	xlabel=$x$, 
	ylabel=$S(x)$,          
	samples=100,
	clip=true,
	]
	\addplot[color=black, line width=1.5pt,domain=-9:-6.28] {(x+9.42)/2};	
	\addplot[color=black, line width=1.5pt,domain=-6.28:0] {(x+3.14)/2};
	\addplot[color=black, line width=1.5pt,domain=0:6.28]{(x-3.14)/2};
	\addplot[color=black, line width=1.5pt,domain=6.28:9]{(x-9.42)/2};
	\addplot[thick,dashed] coordinates {(-6.28,-1.57) (-6.28,1.57)};
	\addplot[thick,dashed] coordinates {(0,-1.57) (0,1.57)};
	\addplot[thick,dashed] coordinates {(6.28,-1.57) (6.28,1.57)};
	\addplot[
	mark=*,
	mark options={fill=black, draw=black},
	only marks,
	] coordinates {(-6.28, 0)  (0, 0) (6.28, 0)};
	\end{axis}
	\end{tikzpicture}
\end{center}
\noindent
\textbf{Ответ:}
\[
f(x)=\sum_{n=1}^\infty\left( \frac{(-1)^n-1}{n}-\frac{(-1)^n}{n} \right )\sin(nx), x\ne 2\pi n;
 \text{ }S(2\pi n)=0, \text{ при } n\in\mathbb{Z}.
\]




% ---------------------------- Problem 2----------------------------------
\subsubsection*{\center Задача № 2.}
{\bf Условие.~}
Для заданной графически функции $y(x)$ построить ряд Фурье в комплексной форме, изобразить график суммы построенного ряда

%График
\begin{center}
	\begin{tikzpicture}
	\begin{axis}[xmin=-1,	xmax=4.5, 	ymin=-0.5,	ymax=0.5,
	width=0.5\textwidth,
	height=0.4\textwidth,
	axis x line=middle,
	axis y line=middle, 
	every axis x label/.style={at={(current axis.right of origin)},anchor=west},
	every inner x axis line/.append style={|-latex'},
	every inner y axis line/.append style={|-latex'},
	minor tick num=1,			
	axis equal=true,
	xlabel=$x$, 
	ylabel=$y$,          
	samples=100,
	clip=true,
	]
	\addplot[color=black, line width=1.5pt,domain=0:2] {x/2-1};
	\addplot[color=black, line width=1.5pt,domain=2:4]{1};
	\addplot[thick,dashed] coordinates {(2,0) (2,1)};
	\addplot[thick,dashed] coordinates {(4,0) (4,1)};
	\end{axis}
	\end{tikzpicture}
\end{center}

\noindent
\textbf{Решение.}\\

\noindent
Ряд Фурье в комплексной форме имеет следующий вид
\[
f(x) = \sum_{n=-\infty}^\infty c_n e^{i\omega nx},\quad c_n=\frac{1}{T}\int\limits_a^b f(x) e^{-i\omega nx}dx,\,\omega=\frac{2\pi}{T}.
\]
В нашем примере $ a=0,b=4,T=4,\omega=\pi/2$, 
найдем коэффицинеты $c_n,\,n=0,\pm1,\pm2,\ldots$
где $\omega=2\pi/T,\,T=4.$
$$
\begin{array}{rcl}
c_0 &=&\displaystyle\frac{1}{4} \int\limits_0^4 f(x)dx=\frac{1}{4}\left( \int\limits_0^2 (\frac{x}{2}-1)dx + \int\limits_2^4 dx \right)=\frac{1}{4},\\[12pt]
c_n &=&\displaystyle\frac{1}{4}\left(
\int\limits_0^2
(\frac{x}{2}-1)e^{-i\omega nx}dx + \int\limits_2^4
e^{-i\omega nx}dx \right) ={}\\[12pt]
&=&\displaystyle\frac{1}{4}\left(
-\left.\frac{2ie^{-i\omega nx}}{\pi n}\right|_0^2
+\left.\frac{e^{-i\omega nx}}{\pi n}\left(\frac{2}{\pi n} + {ix}\right)\right|_0^2 + \left.\frac{2ie^{-i\omega nx}}{\pi n}\right|_2^4\right) = \\[12pt]
&=&\displaystyle\frac{i(-e^{-i\pi n}+1+e^{-2i\pi n})}{2\pi n}-\frac{1-e^{-i\pi n}}{2\pi^2 n^2}=\frac{(-1)^n-2}{2 i\pi n}+\frac{(-1)^n-1}{2\pi^2 n^2}.
\end{array}
$$
\noindent
Применив теорему Дирихле о поточечной сходимости ряда Фурье, видим, что построенный ряд Фурье сходится 
к периодическому (с периодом $T=4$) продолжению исходной функции при всех $x\ne 4n-2$ и при $x\ne 4n$, $S(4n-2)=1/2$ и $S(4n)=0$ при 
$n=0,\pm1,\pm2,\ldots$, где $S(x)$ --- сумма ряда Фурье. График функции $S(x)$ имеет вид
\begin{center}
	\begin{tikzpicture}
	\begin{axis}[xmin=-8, xmax=8, ymin=-1, ymax=0.5,
	width=0.8\textwidth,
	height=0.4\textwidth,
	axis x line=middle,
	axis y line=middle, 
	every axis x label/.style={at={(current axis.right of origin)},anchor=west},
	every inner x axis line/.append style={|-latex'},
	every inner y axis line/.append style={|-latex'},
	minor tick num=1,			
	axis equal=true,
	xlabel=$x$, 
	ylabel=$S(x)$,          
	samples=100,
	clip=true,
	]
	\addplot[color=black, line width=1.5pt,domain=-8:-6] {x/2+3};
	\addplot[color=black, line width=1.5pt,domain=-6:-4]{1};
	\addplot[color=black, line width=1.5pt,domain=-4:-2] {x/2+1};
	\addplot[color=black, line width=1.5pt,domain=-2:0]{1};
	\addplot[color=black, line width=1.5pt,domain=0:2]{x/2-1};
	\addplot[color=black, line width=1.5pt,domain=2:4] {1};
	\addplot[color=black, line width=1.5pt,domain=4:6]{x/2-3};
    \addplot[color=black, line width=1.5pt,domain=6:8] {1};
	\addplot[thick,dashed] coordinates {(-6,0) (-6,1)};
	\addplot[thick,dashed] coordinates {(-4,-1) (-4,1)};
	\addplot[thick,dashed] coordinates {(-2,0) (-2,1)};
	\addplot[thick,dashed] coordinates {(0,-1) (-0,1)};
	\addplot[thick,dashed] coordinates {(2,0) (2,1)};
	\addplot[thick,dashed] coordinates {(4,-1) (4,1)};
	\addplot[thick,dashed] coordinates {(6,0) (6,1)};
	\addplot[
	mark=*,
	mark options={fill=black, draw=black},
	only marks,
	] coordinates {(-6, 0.5) (-4, 0) (-2, 0.5) (0, 0) (2, 0.5) (4, 0) (6, 0.5)};
	\end{axis}
	\end{tikzpicture}
\end{center}

\noindent
\textbf{Ответ:}
\[
\begin{split}
&f(x)=\sum_{n=-\infty}^\infty\left[ \frac{(-1)^n-2}{2 i\pi n}+\frac{(-1)^n-1}{2\pi^2 n^2}\right] e^{\tfrac{i\pi nx}{2}},~ x\ne 4n, 4n-2; \\
&S(4n)=0,\quad S(4n-2)=1/2\quad\text{при}~n\in\mathbb{Z}.
\end{split}
\]


% ---------------------------- Problem 3----------------------------------
\subsubsection*{\center Задача № 3.}
{\bf Условие.~}\\
Найти резольвенту для интегрального уравнения Вольтерры со следующим ядром 
$$K(x,t)=\frac{t^2+t+1}{x^2+x+1}.$$

\noindent
{\bf Решение.~}\\
\noindent
Запишем интегральное уравнение Вольтерры 2-го рода:
$$
y(x)=f(x) + \lambda\int\limits_a^x K(x,t)y(t)dt.
$$
Из рекурентных соотношений получаем
$$
\begin{array}{rcl}
K_1(x,t)&=&\displaystyle \frac{t^2+t+1}{x^2+x+1}, \\[12pt]
K_2(x,t)&=&\displaystyle\int\limits_t^x K(x,s)K_1(s,t)ds = \int\limits_t^x\frac{s^2+s+1}{x^2+x+1}\cdot \frac{t^2+t+1}{s^2+s+1}ds = (x-t)\frac{t^2+t+1}{x^2+x+1},\\[12pt]
K_3(x,t)&=&\displaystyle\int\limits_t^x K(x,s)K_2(s,t)ds = \int\limits_t^x\frac{s^2+s+1}{x^2+x+1}\cdot (s-t)\frac{t^2+t+1}{s^2+s+1} ds =\frac{(x-t)^2}{2}\frac{t^2+t+1}{x^2+x+1}.\\[12pt]
K_j(x,t)&=&\displaystyle\frac{t^2+t+1}{x^2+x+1}\cdot\frac{(x-t)^{j-1}}{(j-1)!},\quad j\in\mathbb{N}.
\end{array}
$$
Подставляя это выражение для итерированных ядер, найдем резольвенту
$$ 
R(x,t,\lambda)=\frac{t^2+t+1}{x^2+x+1}\sum_{j=1}^\infty\lambda^{j-1}\cdot\frac{(x-t)^{j-1}}{(j-1)!},
\quad j=1,2,\ldots
$$